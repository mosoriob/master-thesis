\chapter{Introducción}
\label{Chapter1} 
\newcommand{\keyword}[1]{\textbf{#1}}
\newcommand{\tabhead}[1]{\textbf{#1}}
\newcommand{\code}[1]{\texttt{#1}}
\newcommand{\file}[1]{\texttt{\bfseries#1}}
\newcommand{\option}[1]{\texttt{\itshape#1}}

Científicos han argumentado sobre la importancia de poder reproducir experimentos científicos debido a la necesidad de corroborar los resultados y conclusiones y a la necesidad de contribuir en el mismo trabajo. 

Sin embargo, proceso de reproducibilidad puede ser tan complejo y costoso que ha sido referido como una tarea forense \cite{baggerly2009deriving}
De hecho, estudios controversiales y escándalos han expuesto la necesidad de mejorar los procesos de reproducibilidad de las investigaciones. Algunos ejemplos son: casos clínicos \cite{ioannidis2009repeatability}, sondeos del área de psicología \cite{open2015estimating} o estudios en área de la computación \cite{baggerly2009deriving}

La reproducibilidad del experimento es la capacidad de volver a ejecutar un experimento con o sin la introducción de cambios en él y la obtención de resultados que sean consistentes con los originales.
La introducción de cambios permite evaluar diferentes características de ese experimento ya que los investigadores pueden modificarlo gradualmente, mejorando y re orientando los métodos y condiciones experimentales~\cite{stodden2010reproducible}.
%Comienza mostrar las necesidades de la reproducibilidad y introducción a los workflows

Para permitir la reproducibilidad del experimento es necesario proporcionar suficiente información sobre él mismo, permitiendo comprenderlo, evaluarlo y volver a construirlo. Normalmente, los experimentos se describen en flujos de trabajo científicos (representaciones que permiten gestionar cálculos a gran escala) que se ejecutan en sistemas informáticos distribuidos. 
%Aparece el problema de la conservación de los experimentos
Para permitir la reproducibilidad de estos flujos de trabajos científicos es necesario, en primer lugar, abordar un problema de conservación del flujo de trabajo. 
La conservación es un acto de proporcionar información compresible que describa el contexto original del experimento ~\cite{conservation} y es por ello que los flujos de trabajo experimentales necesitan garantizar que haya suficiente información sobre los experimentos para que sea posible construirlos nuevamente por un tercero, replicando sus resultados sin ninguna información adicional del autor original ~\cite{garijo2013quantifying}. 

%Cual es la conversación que se hace ahora
Para alcanzar la conservación la comunidad se ha enfocado en la conservación de las ejecuciones de workflow conservando los datos, código y la descripción,  pero no dejando de lado la infraestructura subyacente (e.g, recursos computacionales y componentes de software).
Existe otros enfoques que se han enfocado en conservar el ambiente computacional de un experimento como WICUS~\cite{santana2017reproducibility} o el proyecto TIMBUS~\footnote{\url{http://www.timbusproject.net/}}~\cite{dappert2013describing} que describe los procesos de negocios y el software y el hardware subyacente.

%todo: añadir información sobre maquinas virtuales

%Hablamos los tipos de conservación que existe
Los autores en~\cite{santana2017reproducibility} identificaron dos enfoques para conservar el ambiente computacional de un experimento científico: la conservación física, en la que los objetos de investigación dentro del experimento se conservan en un entorno virtual (e.g., máquinas virtuales); y la conservación lógica, en la que se describen las principales capacidades de los recursos del ambiente computacional utilizando vocabularios semánticos para que el investigador pueda reproducir un entorno equivalente. Definieron un proceso para documentar la aplicación de flujo de trabajo y su sistema de gestión relacionado, así como sus dependencias. 

Sin embargo, este proceso es realizado en forma manual, dejando trabajo restante para los científicos. Además, al igual que la mayoría de los trabajos dejan fuera del alcance la conservación física del entorno computacional del flujo de trabajo (basándose en la infraestructura elegida). Sin embargo, la conservación lógica y física es importante para lograr la reproducibilidad del experimento. 

El problema de reproducibilidad no sólo sucede en la comunidad científica. Compañías IT también enfrenta problemas similares cuando ellos desean distribuir cualquier producto de software a múltiples servidores.
Para resolver esto, compañías utilizan virtualización a nivel. Esta tecnología se refiere a una característica del Sistema Operativo (SO) en la que el kernel del SO permite la existencia de múltiples instancias aisladas de espacio de usuario llamadas maquinas virtuales o contenedores. 
Una de las tecnologías de virtualización más populares en la industria es Docker\footnote{\url{https://www.docker.com/}}, que implementa la virtualización de software mediante la creación de versiones mínimas de un sistema operativo base (un contenedor).
Los Docker Containers pueden ser vistos como máquinas virtuales ligeras que permiten el ensamblaje de un entorno computacional, incluyendo todas las dependencias necesarias, como bibliotecas, configuración, código y datos necesarios, entre otros. 


%need
%In order to reproduce computational scientific experiments by other researchers, it is mandatory to first allow scientists to share these experiments and second to allow to execute them again using the same (or a very similar) computational environment. In this ways a scientist will have guarantees that the experiment she is executing is running using the same environment from which it was created. Thus, it is needed a procedure to guarantee both requirements: to preserve both logical and physical environments to re execute data workflows with reproducibility guarantees.
Para reproducir experimentos científicos computacionales de otros investigadores, es obligatorio permitir que los científicos compartan estos experimentos o permitir que los ejecuten de nuevo ambiente computacional reproducido (el mismo o muy similar). De esta manera, un científico tendrá garantías de que el experimento que está ejecutando en el mismo entorno o similar desde el que fue creado.
Por lo tanto, se necesita un procedimiento que garantice ambos requisitos: preservar los entornos tanto lógicos como físicos para volver a ejecutar los flujos de trabajo de datos con garantías de reproducibilidad.
%To allow this reproducibility is necessary thus to provide a way to first describe the scientific experiments and the computational environment in which that experiment was executed. Logical conservation allows to describe the computational environment in which experiment was executed and the physical conservation allows the researchers to rerun an experiment without having to deal with the physical distribution of the experiment and not having to deal with dependencies problems. 
Para permitir esta reproducibilidad es necesario, por lo tanto, proporcionar una manera de describir primero los experimentos científicos y el entorno computacional en el que se ejecutó el experimento. La conservación lógica permite describir el entorno computacional en el que se ejecutó el experimento y la conservación física permite a los investigadores volver a ejecutar un experimento sin tener que ocuparse de la distribución física del a y sin tener que ocuparse de los problemas de dependencias. 



Por lo tanto se una solución para mejorar la solución de conservación física y lógica mediante el uso de contenedores. 
Se propone en primer lugar utilizar las imágenes Docker como medio para preservar el entorno físico de un experimento. Se utiliza contenedores ya que son ligeros y, lo que es más importante, son más fáciles de describir automáticamente, por lo que mejoramos el proceso de documentación de los flujos de trabajo científicos.
Con Docker, los usuarios pueden distribuir estos entornos computacionales a través de imágenes de software utilizando un repositorio público llamado DockerHub. Por lo tanto, nuestro objetivo es abordar la conservación física.
Con el fin de lograr una conservación lógica, se construye un sistema de anotador para las imágenes Docker que describe el sistema de gestión del flujo de trabajo, así como sus dependencias mediante el desarrollo de un sistema de anotador para las imágenes Docker antes. De esta manera, se pretende abordar la conservación física.
Para validar la solución se reproducen cuatros experimentos computacionales diferentes. Estos experimentos abarcan diferentes sistemas, lenguajes y configuraciones, lo que demuestra que nuestro enfoque es genérico y puede aplicarse a cualquier experimento computacional. Realizamos estos experimentos, los describimos lógicamente y a continuación los reproducimos en base a las descripciones lógicas que obtuvimos anteriormente. Para validar el enfoque, comparamos los resultados de los experimentos.

%\subsection{Experimental Sciences Approaches}

%\subsubsection{In Vivo and In Vitro Science}

%\subsubsection{In Silico Science}

%\subsubsection{The Challenge of Scientific Reproducibility}

%\subsection{El desafio de la reproducibilidad}
%
%El problema de la reproducibilidad entre distintos tipos de practicas cientificas varia entre una y otra.
%Por ejemplo, la biología los recursos son muestras de celulas u otro
%
%La utilización de equipamiento también es una limitación
%
%En el caso de experimentos in-silico estas limitaciones no son tan fuertes debido a que los sistemas digitales pueden compartirse y replicar. 
%
%Los experimentos computaciones descansan en artefactos digitales para resolver sus 
%
%
%Sistemas computacionales se basan sistemas digitales que son almacenados
%
%En este trabajo, se busca soluciona el este problema, para ello se 
%
%mas
%
%En ee
%
%
%\section{Experimentos científicos computacionales} 
%
%La reproducibilidad de los resultados en experimentos es una piedra angular del método científico. Es por ello, que la comunidad científica ha incentivado a los investigadores a publicar sus contribuciones en un forma verificable y entendible \cite{james2014standing,stodden2010reproducible}.
%
%Los términos de reproducibilidad y repetibilidad son utilizados como sinónimos. En este trabajo, utilizaremos las definiciones propuestas por \cite{santana2017reproducibility}, replicabilidad será definida como recreación estricta del experimento original y reproducibilidad es menos restrictivo e implica que puedan existen algunos cambios.
%
%Particularmente, en las áreas de la ciencia donde se ejecutan experimentos in-silico, o sea que realizan a través de computadores o simulaciones, la reproducibilidad requiere que los investigadores compartan el código y los datos de los experimentos realizados con el fin de que tanto los resultados como el método pueden ser analizados en una forma similar al trabajo original descrito en la publicación asociada a dicho experimento. Para lograr ese objetivo, el código debe estar disponible y los datos deben encontrarse en un formato leíble \cite{stodden2014implementing}.
%
%
%
%
%\subsection{Computational Scientific Experiments}
%
%\subsubsection{Workflows in Science}
%
%Loas
%
%conservación fisica y lógica
%
%\subsection{Scientific Conservation and Reproducibility}
%
%\section{Estructura de tesis}
