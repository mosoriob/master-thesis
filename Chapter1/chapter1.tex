% Chapter 1

\chapter{Introducción} % Main chapter title

\label{Chapter1} % For referencing the chapter elsewhere, use \ref{Chapter1} 

%----------------------------------------------------------------------------------------

% Define some commands to keep the formatting separated from the content 
\newcommand{\keyword}[1]{\textbf{#1}}
\newcommand{\tabhead}[1]{\textbf{#1}}
\newcommand{\code}[1]{\texttt{#1}}
\newcommand{\file}[1]{\texttt{\bfseries#1}}
\newcommand{\option}[1]{\texttt{\itshape#1}}

Experiment reproducibility is the ability to run an experiment with the introduction of changes to it and getting results that are consistent with the original ones. Introducing changes allows to evaluate different experimental features of that experiment since researchers can incrementally modify it, improving and re-purposing the experimental methods and conditions~\cite{stodden2010reproducible}.
%Comienza mostrar las necesidades de la reproducibilidad y introducción a los workflows

To allow experiment reproducibility it is necessary to provide enough information about that experiment, allowing to understand, evaluate and build it again. Usually, experiments are described in scientific workflows (representations that allow managing large scale computations) which run on distributed computing systems. 

%Aparece el problema de la conservación de los experimentos
To allow reproducibility of these scientific workflows it is necessary first to address a workflow conservation problem. Conservation refers to the fact of obtaining the same result from an experiment in a different environment~\cite{conservation}. Experimental workflows need to guarantee that there is enough information about the experiments so it is possible to build them again by a third party, replicating its results without any additional information from the original author~\cite{garijo2013quantifying}. 

%Cual es la conversación que se hace ahora
To achieve conservation the research community has focused on conserving workflow executions by conserving data, code, and the workflow description, but not the underlying infrastructure (i.e. computational resources and software components). There are some approaches that focused on conserving the environment of an experiment such as the work in~\cite{santana2017reproducibility} or the Timbus project\footnote{\url{http://www.timbusproject.net/}}~\cite{dappert2013describing} that focuses on business processes and the underlying software and hardware infrastructure.

%Hablamos los tipos de conservación que existen
The authors in~\cite{santana2017reproducibility} identified two approaches for conserving the environment of a scientific experiment: physical conservation, where the research objects within the experiment are conserved in a virtual environment; and logical conservation, where the main capabilities of the resources in the environment are described using semantic vocabularies to allow a researcher to reproduce an equivalent setting. They defined a process for documenting the workflow application and its related management system, as well as their dependencies. 

%Introducimos la necesidad
However, this process is done in a semi-automated manner, leaving much work left to the scientists. Furthermore, usually most works leave out of the scope the physical conservation of the workflow computational environment (relying on the chosen infrastructure). However, logical and physical conservation are important to achieve experiment reproducibility. 

The reproducibility problem not only happens in the scientific community. IT-companies also face similar problems when they want to distribute any software product into several hosts. To solve it, companies use operating-system-level virtualization. This technology, also known as containerization, refers to an Operating System (OS) feature in which the OS kernel allows the existence of multiple isolated user-space instances called containers. 
One of the most popular virtualization technologies is Docker\footnote{\url{https://www.docker.com/}}, which implements software virtualization by creating minimal versions of a base operating system (a container). 
Docker Containers can be seen as lightweight virtual machines that allow the assembling of a computational environment, including all necessary dependencies, e.g., libraries, configuration, code and data needed, among others. 

%need
In order to reproduce computational scientific experiments by other researchers, it is mandatory to first allow scientists to share these experiments and second to allow to execute them again using the same (or a very similar) computational environment. In this ways a scientist will have guarantees that the experiment she is executing is running using the same environment from which it was created. Thus, it is needed a procedure to guarantee both requirements: to preserve both logical and physical environments to re execute data workflows with reproducibility guarantees.

To allow this reproducibility is necessary thus to provide a way to first describe the scientific experiments and the computational environment in which that experiment was executed. Logical conservation allows to describe the computational environment in which experiment was executed and the physical conservation allows the researchers to rerun an experiment without having to deal with the physical distribution of the experiment and not having to deal with dependencies problems. 

Herein we propose a solution to improve the physical and logical conservation solution by using containers. 
We propose first to use Docker images as means for preserving the physical environment of an experiment. We use containers since they are lightweight and more importantly, they are easier to automatically describe so we improve the process of documenting scientific workflows.
Using Docker, the users can distribute these computational environments through software images using a public repository called DockerHub. Thus, we aim to address the physical conservation.
In order to achieve logical conservation, we built an annotator system for the Docker Images that describe the workflow management system, as well as their dependencies by developing an annotator system for the Docker images before. In this way, we aim to address the physical conservation.
To validate our solution we reproduce 4 different computational experiments. These experiments span different systems, languages and configurations, showing that our approach is generic and can be applied to any computational experiment. We run these experiments, we describe them logically and next we reproduce them based on the logical descriptions we obtained before. To validate the approach we compare the outputs from the experiments.



%\subsection{Experimental Sciences Approaches}

%\subsubsection{In Vivo and In Vitro Science}

%\subsubsection{In Silico Science}

%\subsubsection{The Challenge of Scientific Reproducibility}

\subsection{El desafio de la reproducibilidad}

El problema de la reproducibilidad entre distintos tipos de practicas cientificas varia entre una y otra.
Por ejemplo, la biología los recursos son muestras de celulas u otro

La utilización de equipamiento también es una limitación

En el caso de experimentos in-silico estas limitaciones no son tan fuertes debido a que los sistemas digitales pueden compartirse y replicar. 

Los experimentos computaciones descansan en artefactos digitales para resolver sus 


Sistemas computacionales se basan sistemas digitales que son almacenados

En este trabajo, se busca soluciona el este problema, para ello se 

mas

En ee


\section{Experimentos científicos computacionales} 

La reproducibilidad de los resultados en experimentos es una piedra angular del método científico. Es por ello, que la comunidad científica ha incentivado a los investigadores a publicar sus contribuciones en un forma verificable y entendible \cite{james2014standing,stodden2010reproducible}.

Los términos de reproducibilidad y repetibilidad son utilizados como sinónimos. En este trabajo, utilizaremos las definiciones propuestas por \cite{santana2017reproducibility}, replicabilidad será definida como recreación estricta del experimento original y reproducibilidad es menos restrictivo e implica que puedan existen algunos cambios.

Particularmente, en las áreas de la ciencia donde se ejecutan experimentos in-silico, o sea que realizan a través de computadores o simulaciones, la reproducibilidad requiere que los investigadores compartan el código y los datos de los experimentos realizados con el fin de que tanto los resultados como el método pueden ser analizados en una forma similar al trabajo original descrito en la publicación asociada a dicho experimento. Para lograr ese objetivo, el código debe estar disponible y los datos deben encontrarse en un formato leíble \cite{stodden2014implementing}.




\subsection{Computational Scientific Experiments}

\subsubsection{Workflows in Science}

Loas

conservación fisica y lógica

\subsection{Scientific Conservation and Reproducibility}

\section{Estructura de tesis}
